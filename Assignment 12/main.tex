%%%%%%%%%%%%%%%%%%%%%%%%%%%%%%%%%%%%%%%%%%%%%%%%%%%%%%%%%%%%%%%
%
% Welcome to Overleaf --- just edit your LaTeX on the left,
% and we'll compile it for you on the right. If you open the
% 'Share' menu, you can invite other users to edit at the same
% time. See www.overleaf.com/learn for more info. Enjoy!
%
%%%%%%%%%%%%%%%%%%%%%%%%%%%%%%%%%%%%%%%%%%%%%%%%%%%%%%%%%%%%%%%

% Inbuilt themes in beamer
\documentclass{beamer}

%packages:
% \usepackage{tfrupee}
% \usepackage{amsmath}
% \usepackage{amssymb}
% \usepackage{gensymb}
% \usepackage{txfonts}

% \def\inputGnumericTable{}

% \usepackage[latin1]{inputenc}                                 
% \usepackage{color}                                            
% \usepackage{array}                                            
% \usepackage{longtable}                                        
% \usepackage{calc}                                             
% \usepackage{multirow}                                         
% \usepackage{hhline}                                           
% \usepackage{ifthen}
% \usepackage{caption} 
% \captionsetup[table]{skip=3pt}  
% \providecommand{\pr}[1]{\ensuremath{\Pr\left(#1\right)}}
% \providecommand{\cbrak}[1]{\ensuremath{\left\{#1\right\}}}
% %\renewcommand{\thefigure}{\arabic{table}}
% \renewcommand{\thetable}{\arabic{table}}      

\setbeamertemplate{caption}[numbered]{}

\usepackage{enumitem}
\usepackage{tfrupee}
\usepackage{amsmath}
\usepackage{amssymb}
\usepackage{gensymb}
\usepackage{graphicx}
\usepackage{txfonts}

\def\inputGnumericTable{}

\usepackage[latin1]{inputenc}                                 
\usepackage{color}                                            
\usepackage{array}                                            
\usepackage{longtable}                                        
\usepackage{calc}                                             
\usepackage{multirow}                                         
\usepackage{hhline}                                           
\usepackage{ifthen}
\usepackage{caption} 
\captionsetup[table]{skip=3pt}  
\providecommand{\pr}[1]{\ensuremath{\Pr\left(#1\right)}}
\providecommand{\cbrak}[1]{\ensuremath{\left\{#1\right\}}}
\newcommand{\myvec}[1]{\ensuremath{\begin{pmatrix}#1\end{pmatrix}}}
\renewcommand{\thefigure}{\arabic{table}}
\renewcommand{\thetable}{\arabic{table}}   
\providecommand{\brak}[1]{\ensuremath{\left(#1\right)}}

% Theme choice:
\usetheme{CambridgeUS}

% Title page details: 
\title{Assignment 11} 
\author{Govinda Rohith Y\\CS21BTECH11062}
\date{\today}
\logo{\large \LaTeX{}}


\begin{document}

% Title page frame
\begin{frame}
    \titlepage 
\end{frame}

% Remove logo from the next slides
\logo{}


% Outline frame
\begin{frame}{Outline}
    \tableofcontents
\end{frame}


% Lists frame
\section{Question}
\begin{frame}{Question}

\begin{block}{\textbf{6-55(Papoullis):}}
         Let $X$ represent the number of successes and $Y$ the number of failures of n independent
Bernoulli trials with p representing the probability of success in anyone trial. Find the distribution of $Z =X -Y$. Show that $E\cbrak{z} = n(2p - 1), Var\cbrak{Z} = 4np(1 - p). $
    \end{block}

\end{frame}
\section{Solution(a)}
\begin{frame}{Solution(a)}
    \begin{block}{Solution(a)}
    Let 
    \begin{align}
        X=k\\
        \implies Y=n-k\\
        Z=X-Y\\
        \implies \boxed{Z=2X-n}
    \end{align}
    So the Z can take values $\cbrak{-n,-(n-2),\cdots,n}$\\
    $\implies Z$ is Binomial distribution.
    \end{block}
\end{frame}
\begin{frame}{Solution (a)}
    \begin{block}{Contd.}
    \begin{align}
        \pr{\{ Z=z \}}=\pr{\{2X-n=z\}}\pr{\{X=\frac{n+z}{2}\}}\\
        \pr{\{Z=z\}}=\myvec{n\\n+z/2}p^{{(n+z)}/2} q^{{(n-z)}/2}
    \end{align}
    \end{block}
\end{frame}
\section{Section(b)}
\begin{frame}{Solution (b)}
    \begin{block}{Solution(b)}
    \begin{align}
        E(Z)=E(2X-n)\\
        E(Z)=2np-n\\
        \implies\boxed{E(Z)=n(2p-1)}
    \end{align}
    \end{block}
\end{frame}
\section{Solution(c)}
\begin{frame}{Solution(c)}
    \begin{block}{Solution(c)}
    \begin{align}
        Var(Z)=E((Z-\mu_Z^2)^2)\\
        Var(Z)=4E((X-np)^2)\\
        Var(Z)=4Var(x)\\
        \implies\boxed{Var(Z)=4npq}
    \end{align}
    
    \end{block}
\end{frame}
\end{document}