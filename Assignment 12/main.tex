%%%%%%%%%%%%%%%%%%%%%%%%%%%%%%%%%%%%%%%%%%%%%%%%%%%%%%%%%%%%%%%
%
% Welcome %%%%%%%%%%%%%%%%%%%%%%%%%%%%%%%%%%%%%%%%%%%%%%%%%%%%%%%%%%%%%%%
%
% Welcome to Overleaf --- just edit your LaTeX on the left,
% and we'll compile it for you on the right. If you open the
% 'Share' menu, you can invite other users to edit at the same
% time. See www.overleaf.com/learn for more info. Enjoy!
%
%%%%%%%%%%%%%%%%%%%%%%%%%%%%%%%%%%%%%%%%%%%%%%%%%%%%%%%%%%%%%%%

% Inbuilt themes in beamer
\documentclass{beamer}

%packages:
% \usepackage{tfrupee}
% \usepackage{amsmath}
% \usepackage{amssymb}
% \usepackage{gensymb}
% \usepackage{txfonts}

% \def\inputGnumericTable{}

% \usepackage[latin1]{inputenc}                                 
% \usepackage{color}                                            
% \usepackage{array}                                            
% \usepackage{longtable}                                        
% \usepackage{calc}                                             
% \usepackage{multirow}                                         
% \usepackage{hhline}                                           
% \usepackage{ifthen}
% \usepackage{caption} 
% \captionsetup[table]{skip=3pt}  
% \providecommand{\pr}[1]{\ensuremath{\Pr\left(#1\right)}}
% \providecommand{\cbrak}[1]{\ensuremath{\left\{#1\right\}}}
% %\renewcommand{\thefigure}{\arabic{table}}
% \renewcommand{\thetable}{\arabic{table}}      

\setbeamertemplate{caption}[numbered]{}

\usepackage{enumitem}
\usepackage{tfrupee}
\usepackage{amsmath}
\usepackage{amssymb}
\usepackage{gensymb}
\usepackage{graphicx}
\usepackage{txfonts}

\def\inputGnumericTable{}

\usepackage[latin1]{inputenc}                                 
\usepackage{color}                                            
\usepackage{array}                                            
\usepackage{longtable}                                        
\usepackage{calc}                                             
\usepackage{multirow}                                         
\usepackage{hhline}                                           
\usepackage{ifthen}
\usepackage{caption} 
\captionsetup[table]{skip=3pt}  
\providecommand{\pr}[1]{\ensuremath{\Pr\left(#1\right)}}
\providecommand{\cbrak}[1]{\ensuremath{\left\{#1\right\}}}
\providecommand{\abs}[1]{\left\vert#1\right\vert}
\newcommand{\myvec}[1]{\ensuremath{\begin{pmatrix}#1\end{pmatrix}}}
\renewcommand{\thefigure}{\arabic{table}}
\renewcommand{\thetable}{\arabic{table}}   
\providecommand{\brak}[1]{\ensuremath{\left(#1\right)}}

% Theme choice:
\usetheme{CambridgeUS}

% Title page details: 
\title{Assignment 12} 
\author{Govinda Rohith Y\\CS21BTECH11062}
\date{\today}
\logo{\large \LaTeX{}}


\begin{document}

% Title page frame
\begin{frame}
    \titlepage 
\end{frame}

% Remove logo from the next slides
\logo{}


% Outline frame
\begin{frame}{Outline}
    \tableofcontents
\end{frame}


% Lists frame
\section{Question}
\begin{frame}{Question}

\begin{block}{\textbf{7-32(Papoullis):}}
         The random variables $X$ and $Y$ are uncorrelated with zero mean and $\sigma_x=\sigma_y=\sigma$. Show that
if$z=x+iy$ then
$$f_z(Z)=f(x,y)=\frac{1}{2\pi\sigma^2}e^{-(x^2+y^2)/2\sigma^2}=\frac{1}{\pi\sigma_Z^2}e^{-\abs{Z^2}/\sigma_z^2}$$
$$\Phi_Z{(\Omega)}=\exp{\left\{-\frac{1}{2}(\sigma^2u^2+\sigma^2v^2_Z)\right\}}=\exp{\left\{-\frac{1}{4}\sigma_Z^2\abs{\Omega}^2\right\}}$$
    \end{block}
\section{Solution(a)}
\end{frame}
\begin{frame}{Solution(a)}
    Since $X$ and $Y$ are independent
    \begin{align}
        \text{Variance of } Z&=X+iY\\
        Var(Z)&=E(\abs{Z-E(Z)^2})=E(\abs{Z}^2)-E(\abs{Z})^2\\
      M_Z(s)=E(e^{-sZ})&=E(e^{-s(x+iY)})\\
      E(e^{-sZ})&=E(e^{-sX})E(e^{-siY})\\
      \implies E(\abs{Z})&=E(X)+iE(Y)=0
    \end{align}
\end{frame}
\section{Solution(a) contd}
\begin{frame}{Contd.}
    \begin{align}
        M_{\abs{Z}^2}(s)&=E(e^{-s\abs{Z}^2})=E(e^{-sX^2})E(e^{-sY^2})\\
        E(e^{-s\abs{Z}^2})&=\left(1+sE(X^2)+\frac{s^2E(X^4)}{2!}+\ldots\right)\left(1+sE(Y^2)+\frac{s^2E(Y^4)}{2!}+\ldots\right)\\
         E(\abs{Z}^2)&=E(X^2)+E(Y^2)
    \end{align}
    Substitute Equations (8,5) in Equation(2)
 \begin{align}
     f_Z(Z)=\frac{1}{2\pi\sigma^2}e^{-(x^2+y^2)/2\sigma^2}=\frac{1}{\pi\sigma_Z^2}e^{-\abs{Z^2}/\sigma_z^2}
 \end{align}
\end{frame}
\section{Solution(b)}
\begin{frame}{Solution(b)}
    \begin{block}{Solution(b)}
    \begin{align}
        \Phi_Z(\Omega) =\Phi_X(u)\Phi_Y(v)\\
        \Phi_Z(\Omega)=e^{-\frac{1}{2}\sigma^2(u^2+v^2)}\\
        \Phi_Z(\Omega)=e^{-\frac{1}{2}\sigma^2\abs{\Omega^2}}\\
        \implies\boxed{\Phi_Z(\Omega)=e^{-\frac{1}{4}\sigma_Z^2\abs{\Omega^2}}}
    \end{align}
            
    \end{block}
\end{frame}
\end{document}}
