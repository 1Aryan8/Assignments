%%%%%%%%%%%%%%%%%%%%%%%%%%%%%%%%%%%%%%%%%%%%%%%%%%%%%%%%%%%%%%%
%
% Welcome %%%%%%%%%%%%%%%%%%%%%%%%%%%%%%%%%%%%%%%%%%%%%%%%%%%%%%%%%%%%%%%
%
% Welcome to Overleaf --- just edit your LaTeX on the left,
% and we'll compile it for you on the right. If you open the
% 'Share' menu, you can invite other users to edit at the same
% time. See www.overleaf.com/learn for more info. Enjoy!
%
%%%%%%%%%%%%%%%%%%%%%%%%%%%%%%%%%%%%%%%%%%%%%%%%%%%%%%%%%%%%%%%

% Inbuilt themes in beamer
\documentclass{beamer}

%packages:
% \usepackage{tfrupee}
% \usepackage{amsmath}
% \usepackage{amssymb}
% \usepackage{gensymb}
% \usepackage{txfonts}

% \def\inputGnumericTable{}

% \usepackage[latin1]{inputenc}                                 
% \usepackage{color}                                            
% \usepackage{array}                                            
% \usepackage{longtable}                                        
% \usepackage{calc}                                             
% \usepackage{multirow}                                         
% \usepackage{hhline}                                           
% \usepackage{ifthen}
% \usepackage{caption} 
% \captionsetup[table]{skip=3pt}  
% \providecommand{\pr}[1]{\ensuremath{\Pr\left(#1\right)}}
% \providecommand{\cbrak}[1]{\ensuremath{\left\{#1\right\}}}
% %\renewcommand{\thefigure}{\arabic{table}}
% \renewcommand{\thetable}{\arabic{table}}      

\setbeamertemplate{caption}[numbered]{}

\usepackage{enumitem}
\usepackage{tfrupee}
\usepackage{amsmath}
\usepackage{amssymb}
\usepackage{gensymb}
\usepackage{graphicx}
\usepackage{txfonts}

\def\inputGnumericTable{}

\usepackage[latin1]{inputenc}                                 
\usepackage{color}                                            
\usepackage{array}                                            
\usepackage{longtable}                                        
\usepackage{calc}                                             
\usepackage{multirow}                                         
\usepackage{hhline}                                           
\usepackage{ifthen}
\usepackage{caption} 
\captionsetup[table]{skip=3pt}  
\providecommand{\pr}[1]{\ensuremath{\Pr\left(#1\right)}}
\providecommand{\cbrak}[1]{\ensuremath{\left\{#1\right\}}}
\providecommand{\abs}[1]{\left\vert#1\right\vert}
\newcommand{\myvec}[1]{\ensuremath{\begin{pmatrix}#1\end{pmatrix}}}
\renewcommand{\thefigure}{\arabic{table}}
\renewcommand{\thetable}{\arabic{table}}   
\providecommand{\brak}[1]{\ensuremath{\left(#1\right)}}

% Theme choice:
\usetheme{CambridgeUS}

% Title page details: 
\title{Assignment 12} 
\author{Govinda Rohith Y\\CS21BTECH11062}
\date{\today}
\logo{\large \LaTeX{}}


\begin{document}

% Title page frame
\begin{frame}
    \titlepage 
\end{frame}

% Remove logo from the next slides
\logo{}


% Outline frame
\begin{frame}{Outline}
    \tableofcontents
\end{frame}


% Lists frame
\section{Question}
\begin{frame}{Question}

\begin{block}{\textbf{9-31(Papoullis):}}
        Show that if
        $$S=\int_0^{10}x(t)dt \text{ then } E(S^2)=\int_{-10}^{10}(10-\abs{\tau})R_X{(\tau})d\tau$$
        Find the mean and variance of $S$ if $E(x(t))=8$,$R_X{(\tau})=64+10^{-2\abs{\tau}}$
    \end{block}
    \end{frame}
    \section{Solution (a)}
    \begin{frame}{Solution (a)}
        \begin{block}{Solution (a)}
        The moment of $S$ is equal to moments of (Since x(t) is WSS)
        \begin{align}
            Z&=\int_{-5}^{5}x(t)dt\\
            \implies E(S^2)&=\int_{-5}^{5}\int_{-5}^{5}R_X(t_1-t_2)dt_1dt_2\\
            \implies\boxed{E(S^2)=\int_{-10}^{10}(10-\abs{\tau})R_X{(\tau})d\tau}
        \end{align}
        \end{block}
    \end{frame}
    
    \section{Solution (b)}
    \begin{frame}{Solution (b}
        \begin{block}{Solution (b)}
        \begin{align}
        s&=\int_0^{10}x(t)dt\\
        \implies E(s)&=\int_0^{10}E(x(t))dt\\
        \text{Given }E(x(t))&=8\\
        \implies \boxed{E(s)=80}
        \end{align}
        \end{block}
    \end{frame}
    \section{Solution(b)..}
    \begin{frame}{Contd$\ldots$}
        \begin{block}{Contd}
    \begin{align}
        Var(s)&=E(S-E(S)^2)=E(S^2)-E(S)^2\\
        \sigma^2&=2\int_0^{10}(10-\tau)(64+10e^{-2\tau})d\tau-80^2\\
        \implies \boxed{\sigma^2\approx9.5}
    \end{align}
        \end{block}
    \end{frame}
    \end{document}
